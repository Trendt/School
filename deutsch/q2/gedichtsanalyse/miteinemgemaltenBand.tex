\documentclass[12pt]{book}
\title{Gedichtsanalyse - Mit einem gemalten Band(1771)}
\author{Jaro Schmidt}
\date{\today}
\usepackage[left=2cm,right=2cm,top=3cm,bottom=4cm,bindingoffset=1cm,includeheadfoot]{geometry}
\begin {document}
	\maketitle
	
	Die Lyrik des Sturm und Drang handelt sehr oft von der Liebe sowie der Natur, dies ist beides der Fall in dem Frühlingsgedicht "Mit einem gemalten Band" von Johann Wolfgang Goethe, welches im Jahr 1771 veröffentlicht wurde.
	Das Gedicht vergleicht und beschreib diese 2 Aspekte, Liebe und Natur, wobei jedoch die Liebe das hauptsächliche Thema ist.

	Das Gedicht leitet mit dem Frühling ein und damit, wie mit diesem die Natur erblüht. Es geht zu "seiner Liebsten"(Z. 6) und der Liebe über, seine Liebste wird als von Rosen umgeben, sowie als Rose selbst beschrieben (Z.9-10).
	Zum Schluss des Gedichtes wird "das Band", welches ihn mit seiner Liebsten verbindet als kein schwaches Rosenband beschrieben, damit soll zum ausdruck gebracht werden, dass die Liebe sehr stark ist.
	Das Lyrische Werk besteht aus 4 Strophen á 4 Verse, somit 16 Verse insgesamt. Es wird ein Kreuzreimschema verwendet, welches sich durch das gesamte Werk zieht.
	Das Metrum ist von der Taktart ein Tröchäus, wobei im ganzen Gedicht der a-Reim aus Zwei achthebiegen Versen und der b-Reim aus Zwei siebenhebigen Versen besteht.
	Der Satzbau ist somit ziemlich genial, da sogar auf solch eine Kleinigkeit geachtet wurde.
	Das auffälligste und am häufigsten auftrentende Stilmitel ist die Rose als Symbol der Liebe zu seiner Geliebten.
	Die Wortwahl insgesamt ist sehr positiv gehalten, wodurch das Gedichtes sehr ansprechend gestaltet wird.
	Dies fällt besonders durch Worte wie Munterkeit auf.
	Wie bereits anfänglich erwähnt, ist das Hauptthema des Gedichtes von Goethe die Liebe zu seiner Geliebten.
	Wenn man dies auf den Titel "Mit einem gemalten Band", sowie auf die letzte Strophe bezieht, stellt man fest, dass sich das Thema in dem Titel wiederspiegelt.
	Mit einem gemalten Band ist das Rosenband gemeint, welches ihn mit siner Geliebten verbindet (Vgl. Zeile 15). Das Wort gemalt soll womöglich darstellen, dass diese Verbindung von den Personen selbst geschaffen wurde, so wie ein Künstler ein Gemälde malt.

	Die Natur spielt in diesem Gedicht eine große Rolle.
	Sowohl die Rose als Symbol der Liebe, sowie das Rosenband als Verbindung zwischen 2 Menschen, als auch der Frühling als Zeit des erwachens sind Symbole welche sehr schön die Liebe beschreiben können. Dies wurde meiner Meinung nach in diesem Gedicht auch geschafft.

\end {document}
