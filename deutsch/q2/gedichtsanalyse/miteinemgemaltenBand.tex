\documentclass[12pt]{book}
\title{Gedichtsanalyse - Mit einem gemalten Band(1771)}
\author{Jaro Schmidt}
\date{\today}
\usepackage[left=2cm,right=2cm,top=3cm,bottom=4cm,bindingoffset=1cm,includeheadfoot]{geometry}
\begin {document}
	\maketitle
	
	Die Lyrik des Sturm und Drang handelt sehr oft von der Liebe sowie der Natur, dies ist beides der Fall in dem Frühlingsgedicht "Mit einem gemalten Band" von Johann Wolfgang Goethe, welches im Jahr 1771 veröffentlicht wurde.
	Das Gedicht vergleicht und beschreib diese 2 Aspekte, Liebe und Natur, wobei jedoch die Liebe das hauptsächliche Thema ist.

	Das Gedicht leitet mit dem Frühling ein und damit, wie mit diesem die Natur erblüht. Es geht zu "seiner Liebsten"(Z. 6) und der Liebe über, seine Liebste wird als von Rosen umgeben, sowie als Rose selbst beschrieben (Z.9-10).
	Zum Schluss des Gedichtes wird "das Band", welches ihn mit seiner Liebsten verbindet als kein schwaches Rosenband beschrieben, damit soll zum ausdruck gebracht werden, dass die Liebe sehr stark ist.
	Das Lyrische Werk besteht aus 4 Strophen á 4 Verse, somit 16 Verse insgesamt. Es wird ein Kreuzreimschema verwendet, welches sich durch das gesamte Werk zieht.
	Das Metrum ist von der Taktart ein Tröchäus, wobei im ganzen Gedicht der a-Reim aus Zwei achthebiegen Versen und der b-Reim aus Zwei siebenhebigen Versen besteht.
	Der Satzbau ist somit ziemlich genial, da sogar auf solch eine Kleinigkeit geachtet wurde.

	$$ stilmitte $$
	$$ Wortwahl $$

	





\end {document}
